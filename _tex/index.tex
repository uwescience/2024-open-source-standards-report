% Options for packages loaded elsewhere
\PassOptionsToPackage{unicode}{hyperref}
\PassOptionsToPackage{hyphens}{url}
\PassOptionsToPackage{dvipsnames,svgnames,x11names}{xcolor}
%
\documentclass[
  number]{elsarticle}

\usepackage{amsmath,amssymb}
\usepackage{iftex}
\ifPDFTeX
  \usepackage[T1]{fontenc}
  \usepackage[utf8]{inputenc}
  \usepackage{textcomp} % provide euro and other symbols
\else % if luatex or xetex
  \usepackage{unicode-math}
  \defaultfontfeatures{Scale=MatchLowercase}
  \defaultfontfeatures[\rmfamily]{Ligatures=TeX,Scale=1}
\fi
\usepackage{lmodern}
\ifPDFTeX\else  
    % xetex/luatex font selection
\fi
% Use upquote if available, for straight quotes in verbatim environments
\IfFileExists{upquote.sty}{\usepackage{upquote}}{}
\IfFileExists{microtype.sty}{% use microtype if available
  \usepackage[]{microtype}
  \UseMicrotypeSet[protrusion]{basicmath} % disable protrusion for tt fonts
}{}
\makeatletter
\@ifundefined{KOMAClassName}{% if non-KOMA class
  \IfFileExists{parskip.sty}{%
    \usepackage{parskip}
  }{% else
    \setlength{\parindent}{0pt}
    \setlength{\parskip}{6pt plus 2pt minus 1pt}}
}{% if KOMA class
  \KOMAoptions{parskip=half}}
\makeatother
\usepackage{xcolor}
\setlength{\emergencystretch}{3em} % prevent overfull lines
\setcounter{secnumdepth}{5}
% Make \paragraph and \subparagraph free-standing
\makeatletter
\ifx\paragraph\undefined\else
  \let\oldparagraph\paragraph
  \renewcommand{\paragraph}{
    \@ifstar
      \xxxParagraphStar
      \xxxParagraphNoStar
  }
  \newcommand{\xxxParagraphStar}[1]{\oldparagraph*{#1}\mbox{}}
  \newcommand{\xxxParagraphNoStar}[1]{\oldparagraph{#1}\mbox{}}
\fi
\ifx\subparagraph\undefined\else
  \let\oldsubparagraph\subparagraph
  \renewcommand{\subparagraph}{
    \@ifstar
      \xxxSubParagraphStar
      \xxxSubParagraphNoStar
  }
  \newcommand{\xxxSubParagraphStar}[1]{\oldsubparagraph*{#1}\mbox{}}
  \newcommand{\xxxSubParagraphNoStar}[1]{\oldsubparagraph{#1}\mbox{}}
\fi
\makeatother


\providecommand{\tightlist}{%
  \setlength{\itemsep}{0pt}\setlength{\parskip}{0pt}}\usepackage{longtable,booktabs,array}
\usepackage{calc} % for calculating minipage widths
% Correct order of tables after \paragraph or \subparagraph
\usepackage{etoolbox}
\makeatletter
\patchcmd\longtable{\par}{\if@noskipsec\mbox{}\fi\par}{}{}
\makeatother
% Allow footnotes in longtable head/foot
\IfFileExists{footnotehyper.sty}{\usepackage{footnotehyper}}{\usepackage{footnote}}
\makesavenoteenv{longtable}
\usepackage{graphicx}
\makeatletter
\def\maxwidth{\ifdim\Gin@nat@width>\linewidth\linewidth\else\Gin@nat@width\fi}
\def\maxheight{\ifdim\Gin@nat@height>\textheight\textheight\else\Gin@nat@height\fi}
\makeatother
% Scale images if necessary, so that they will not overflow the page
% margins by default, and it is still possible to overwrite the defaults
% using explicit options in \includegraphics[width, height, ...]{}
\setkeys{Gin}{width=\maxwidth,height=\maxheight,keepaspectratio}
% Set default figure placement to htbp
\makeatletter
\def\fps@figure{htbp}
\makeatother

\makeatletter
\@ifpackageloaded{caption}{}{\usepackage{caption}}
\AtBeginDocument{%
\ifdefined\contentsname
  \renewcommand*\contentsname{Table of contents}
\else
  \newcommand\contentsname{Table of contents}
\fi
\ifdefined\listfigurename
  \renewcommand*\listfigurename{List of Figures}
\else
  \newcommand\listfigurename{List of Figures}
\fi
\ifdefined\listtablename
  \renewcommand*\listtablename{List of Tables}
\else
  \newcommand\listtablename{List of Tables}
\fi
\ifdefined\figurename
  \renewcommand*\figurename{Figure}
\else
  \newcommand\figurename{Figure}
\fi
\ifdefined\tablename
  \renewcommand*\tablename{Table}
\else
  \newcommand\tablename{Table}
\fi
}
\@ifpackageloaded{float}{}{\usepackage{float}}
\floatstyle{ruled}
\@ifundefined{c@chapter}{\newfloat{codelisting}{h}{lop}}{\newfloat{codelisting}{h}{lop}[chapter]}
\floatname{codelisting}{Listing}
\newcommand*\listoflistings{\listof{codelisting}{List of Listings}}
\makeatother
\makeatletter
\makeatother
\makeatletter
\@ifpackageloaded{caption}{}{\usepackage{caption}}
\@ifpackageloaded{subcaption}{}{\usepackage{subcaption}}
\makeatother

\ifLuaTeX
  \usepackage{selnolig}  % disable illegal ligatures
\fi
\usepackage[]{natbib}
\bibliographystyle{elsarticle-num}
\usepackage{bookmark}

\IfFileExists{xurl.sty}{\usepackage{xurl}}{} % add URL line breaks if available
\urlstyle{same} % disable monospaced font for URLs
\hypersetup{
  pdftitle={Towards an open-source model for data and metadata standards},
  colorlinks=true,
  linkcolor={blue},
  filecolor={Maroon},
  citecolor={Blue},
  urlcolor={Blue},
  pdfcreator={LaTeX via pandoc}}


\setlength{\parindent}{6pt}
\begin{document}

\begin{frontmatter}
\title{Towards an open-source model for data and metadata standards}
\author[1,2]{Ariel Rokem%
\corref{cor1}%
}
 \ead{arokem@uw.edu} 
\author[3,2]{Vani Mandava%
%
}

\author[4,2]{Nicoleta Cristea%
%
}

\author[3,2]{Anshul Tambay%
%
}

\author[5,2]{Andrew J. Connolly%
%
}


\affiliation[1]{organization={University of Washington, Department of
Psychology},city={Seattle},country={USA},countrysep={,},postcodesep={}}
\affiliation[2]{organization={University of Washington, eScience
Institute},city={Seattle},country={USA},countrysep={,},postcodesep={}}
\affiliation[3]{organization={University of Washington, Scientific
Software Engineering
Center},city={Seattle},country={USA},countrysep={,},postcodesep={}}
\affiliation[4]{organization={University of Washington, Department of
Civil and Environmental
Engineering},city={Seattle},country={USA},countrysep={,},postcodesep={}}
\affiliation[5]{organization={University of Washington, Department of
Astronomy},city={Seattle},country={USA},countrysep={,},postcodesep={}}

\cortext[cor1]{Corresponding author}





        
\begin{abstract}
Progress in machine learning and artificial intelligence promises to
advance research and understanding across a wide range of fields and
activities. In tandem, increased awareness of the importance of open
data for reproducibility and scientific transparency is making inroads
in fields that have not traditionally produced large publicly available
datasets. Data sharing requirements from publishers and funders, as well
as from other stakeholders, have also created pressure to make datasets
with research and/or public interest value available through digital
repositories. However, to make the best use of existing data, and
facilitate the creation of useful future datasets, robust, interoperable
and usable standards need to evolve and adapt over time. The open-source
development model provides significant potential benefits to the process
of standard creation and adaptation. In particular, data and meta-data
standards can use long-standing technical and socio-technical processes
that have been key to managing the development of software, and which
allow incorporating broad community input into the formulation of these
standards. On the other hand, open-source models carry unique risks that
need to be considered. This report surveys existing open-source
standards development, addressing these benefits and risks. It outlines
recommendations for standards developers, funders and other stakeholders
on the path to robust, interoperable and usable open-source data and
metadata standards.
\end{abstract}





\end{frontmatter}
    

\section{Introduction}\label{sec-intro}

Data-intensive discovery has become an important mode of knowledge
production across many research fields and it is having a significant
and broad impact across all of society. This is becoming increasingly
salient as recent developments in machine learning and artificial
intelligence (AI) promise to increase the value of large,
multi-dimensional, heterogeneous data sources. Coupled with these new
machine learning techniques, these datasets can help us understand
everything from the cellular operations of the human body, through
business transactions on the internet, to the structure and history of
the universe. However, the development of new machine learning methods
and data-intensive discovery more generally depends on Findability,
Accessibility, Interoperability and Reusability (FAIR) of data
\citep{Wilkinson2016FAIR} as well as metadata \citep{Musen2022metadata}.
One of the main mechanisms through which the FAIR principles are
promoted is the development of \emph{standards} for data and metadata.
Standards can vary in the level of detail and scope, and encompass such
things as \emph{file formats} for the storage of certain data types,
\emph{schemas} for databases that organize data, \emph{ontologies} to
describe and organize metadata in a manner that connects it to
field-specific meaning, as well as mechanisms to describe
\emph{provenance} of analysis products.

Community-driven development of robust, adaptable and useful standards
draws significant inspiration from the development of open-source
software (OSS) and has many parallels and overlaps with OSS development.
OSS has a long history going back to the development of the Unix
operating system in the late 1960s. Over the time since its inception,
the large community of developers and users of OSS have developed a host
of socio-technical mechanisms that support the development and use of
OSS. For example, the Open Source Initiative (OSI), a non-profit
organization that was founded in the 1990s developed a set of guidelines
for licensing of OSS that is designed to protect the rights of
developers and users. On the technical side, tools such as the Git
Source-code management system support complex and distributed
open-source workflows that accelerate, streamline, and make OSS
development more robust. Governance approaches have been honed to
address the challenges of managing a range of stakeholder interests and
to mediate between large numbers of weakly-connected individuals that
contribute to OSS. When these social and technical innovations are put
together they enable a host of positive defining features of OSS, such
as transparency, collaboration, and decentralization. These features
allow OSS to have a remarkable level of dynamism and productivity, while
also retaining the ability of a variety of stakeholders to guide the
evolution of the software to take their needs and interests into
account.

Data and metadata standards that use tools and practices of OSS
(``open-source standards'' henceforth) reap many of the benefits that
the OSS model has provided in the development of other technologies. The
present report explores how OSS processes and tools have affected the
development of data and metadata standards. The report will survey
common features of a variety of use cases; it will identify some of the
challenges and pitfalls of this mode of standards development, with a
particular focus on cross-sector interactions; and it will make
recommendations for future developments and policies that can help this
mode of standards development thrive and reach its full potential.

\section{Use cases}\label{sec-use-cases}

To understand how OSS development practices affect the development of
data and metadata standards, it is informative to demonstrate this
cross-fertilization through a few use cases. As we will see in these
examples, some fields, such as astronomy, high-energy physics and earth
sciences have a relatively long history of shared data resources from
organizations such as SDSS, CERN, and NASA, while other fields have only
relatively recently become aware of the value of data sharing and its
impact. These disparate histories inform how standards have evolved and
how OSS practices have pervaded their development. It also demonstrates
field-specific limitations on the adoption of OSS tools and practices
that exemplify some of the challenges, which we will explore
subsequently.

\subsection{Astronomy}\label{astronomy}

An early prominent example of a community-driven standard is the FITS
(Flexible Image Transport System) file format standard, which was
developed in the late 1970s and early 1980s \citep{wells1979fits}, and
has been adopted worldwide for astronomy data preservation and exchange.
Essentially every software platform used in astronomy reads and writes
the FITS format. It was developed by observatories in the 1980s to store
image data in the visible and x-ray spectrum. It has been endorsed by
the International Astronomical Union (IAU), as well as funding agencies.
Though the format has evolved over time, ``once FITS, always FITS''.
That is, the format cannot be evolved to introduce changes that break
backward compatibility. Among the features that make FITS so durable is
that it was designed originally to have a very restricted metadata
schema. That is, FITS records were designed to be the lowest common
denominator of word lengths in computer systems at the time. However,
while FITS is compact, its ability to encode a coordinate frame for
pixels, means that data from different observational instruments can be
stored in this format and relationships between data from different
instruments can be defined, rendering manual and error-prone procedures
for conforming images obsolete. Nevertheless, the stability has also
raised some issues as the field continues to adapt to new measurement
methods and the demands of ever-increasing data volumes and complex data
analysis use-case, such as interchange with other data and the use of
complex data bases to store and share data \citep{Scroggins2020-ut}.
Another prominent example of the use of open-source processes to develop
standards in Astronomy is in the tools and protocols developed by the
International Virtual Observatory Alliance (IVOA) and its national
implementations, e.g., in the US Virtual Astronomical
Observatory\citep{Hanisch2015-cu}. The virtual observatories facilitate
discovery and access across observatories around the world and underpin
data discovery in astronomy. The IVOA took inspiration from the
World-Wide Web Consortium (W3C) and adopted its process for the
development of its standards (i.e., Working drafts \(\rightarrow\)
Proposed Recommendations \(\rightarrow\) Recommendations), with
individual standards developed by inter-institutional and international
working groups. One of the outcomes of the coordination effort is the
development of an ecosystem of software tools both developed within the
observatory teams and within the user community that interoperate with
the standards that were adopted by the observatories.

\subsection{High-energy physics (HEP)}\label{high-energy-physics-hep}

Because data collection is centralized, standards to collect and store
HEP data have been established and the adoption of these standards in
data analysis has high penetration \citep{Basaglia2023-dq}. A top-down
approach is taken so that within every large collaboration, standards
are enforced, and this adoption is centrally managed. Access to raw data
is essentially impossible because of its large volume, and making it
publicly available would be technically very difficult. Therefore,
analysis tools are tuned specifically to the standards of the released
data. Incentives to use the standards are provided by funders that
require data management plans that specify how the data is shared (i.e.,
in a standards-compliant manner).

\subsection{Earth sciences}\label{earth-sciences}

The need for geospatial data exchange between different systems began to
be recognized in the 1970s and 1980s, but proprietary formats still
dominated. Coordinated standardization efforts brought the Open
Geospatial Consortium (OGC) establishment in the 1990s, a critical step
towards open standards for geospatial data. The 1990s have also seen the
development of key standards such as the Network Common Data Form
(NetCDF) developed by the University Corporation for Atmospheric
Research (UCAR), and the Hierarchical Data Format (HDF), a set of file
formats (HDF4, HDF5) that are widely used, particularly in climate
research. The GeoTIFF format, which originated at NASA in the late
1990s, is extensively used to share image data. The following two
decades, the 2000s-2020s, brought an expansion of open standards and
integration with web technologies developed by OGC, as well as other
standards such as the Keyhole Markup Language (KML) for displaying
geographic data in Earth browsers. Formats suitable for cloud computing
also emerged, such as the Cloud Optimized GeoTIFF (COG), followed by
Zarr and Apache Parquet for array and tabular data, respectively. In
2006, the Open Source Geospatial Foundation (OSGeo,
\url{https://www.osgeo.org}) was established, demonstrating the
community's commitment to the development of open-source geospatial
technologies. While some standards have been developed in the industry
(e.g., Keyhole Markup Language (KML) by Keyhole Inc., which Google later
acquired), they later became international standards of the OGC, which
now encompasses more than 450 commercial, governmental, nonprofit, and
research organizations working together on the development and
implementation of open standards (\url{https://www.ogc.org}).

\subsection{Neuroscience}\label{neuroscience}

In contrast to the previously-mentioned fields, Neuroscience has
traditionally been a ``cottage industry'', where individual labs have
generated experimental data designed to answer specific experimental
questions. While this model still exists, the field has also seen the
emergence of new modes of data production that focus on generating large
shared datasets designed to answer many different questions, more akin
to the data generated in large astronomy data collection efforts
\citep{Koch2012-ve}. This change has been brought on through a
combination of technical advances in data acquisition techniques, which
now generate large and very high-dimensional/information-rich datasets,
cultural changes, which have ushered in new norms of transparency and
reproducibility, and funding initiatives that have encouraged this kind
of data collection. However, because these changes are recent relative
to the other cases mentioned above, standards for data and metadata in
neuroscience have been prone to adopt many elements of modern OSS
development. Two salient examples in neuroscience are the Neurodata
Without Borders file format for neurophysiology data
\citep{Rubel2022NWB} and the Brain Imaging Data Structure (BIDS)
standard for neuroimaging data \citep{Gorgolewski2016BIDS}. BIDS in
particular owes some of its success to the adoption of OSS development
mechanisms \citep{Poldrack2024BIDS}. For example, small changes to the
standard are managed through the GitHub pull request mechanism; larger
changes are managed through a BIDS Enhancement Proposal (BEP) process
that is directly inspired by the Python programming language community's
Python Enhancement Proposal procedure, which is used to introduce new
ideas into the language. Though the BEP mechanism takes a slightly
different technical approach, it tries to emulate the open-ended and
community-driven aspects of Python development to accept contributions
from a wide range of stakeholders and tap a broad base of expertise.

\subsection{Community science}\label{community-science}

Another interesting use case for open-source standards is
community/citizen science. An early example of this approach is
OpenStreetMap (\url{https://www.openstreetmap.org}), which allows users
to contribute to the project development with code and data and freely
use the maps and other related geospatial datasets. But this example is
not unique. Overall, this approach has grown in the last 20 years and
has been adopted in many different fields. It has many benefits for both
the research field that harnesses the energy of non-scientist members of
the community to engage with scientific data, as well as to the
community members themselves who can draw both knowledge and pride in
their participation in the scientific endeavor. It is also recognized
that unique broader benefits are accrued from this mode of scientific
research, through the inclusion of perspectives and data that would not
otherwise be included. To make data accessible to community scientists,
and to make the data collected by community scientists accessible to
professional scientists, it needs to be provided in a manner that can be
created and accessed without specialized instruments or specialized
knowledge. Here, standards are needed to facilitate interactions between
an in-group of expert researchers who generate and curate data and a
broader set of out-group enthusiasts who would like to make meaningful
contributions to the science. This creates a particularly stringent
constraint on transparency and simplicity of standards. Creating these
standards in a manner that addresses these unique constraints can
benefit from OSS tools, with the caveat that some of these tools require
additional expertise. For example, if the standard is developed using
git/GitHub for versioning, this would require learning the complex and
obscure technical aspects of these system that are far from easy to
adopt, even for many professional scientists.

\section{Opportunities and risks for open-source
standards}\label{sec-challenges}

While open-source standards benefit from the technical and social
aspects of OSS, these tools and practices are associated with risks that
need to be mitigated.

\subsection{Flexibility vs.~Stability}\label{flexibility-vs.-stability}

One of the defining characteristics of OSS is its dynamism and its rapid
evolution. Because OSS can be used by anyone and, in most cases,
contributions can be made by anyone, innovations flow into OSS in a
bottom-up fashion from users/developers. Pathways to contribution by
members of the community are often well-defined: both from the technical
perspective (e.g., through a pull request on GitHub, or other similar
mechanisms), as well as from the social perspective (e.g., whether
contributors need to accept certain licensing conditions through a
contributor licensing agreement) and the socio-technical perspective
(e.g., how many people need to review a contribution, what are the
timelines for a contribution to be reviewed and accepted, what are the
release cycles of the software that make the contribution available to a
broader community of users, etc.). Similarly, open-source standards may
also find themselves addressing use cases and solutions that were not
originally envisioned through bottom-up contributions of members of a
research community to which the standard pertains. However, while this
dynamism provides an avenue for flexibility it also presents a source of
tension. This is because data and metadata standards apply to already
existing datasets, and changes may affect the compliance of these
existing datasets. These existing datasets may have a lifespan of
decades, making continued compatibility crucial. Similarly, analysis
technology stacks that are developed based on an existing version of a
standard have to adapt to the introduction of new ideas and changes into
a standard. Dynamic changes of this sort therefore risk causing a loss
of faith in the standard by a user community, and migration away from
the standard. Similarly, if a standard evolves too rapidly, users may
choose to stick to an outdated version of a standard for a long time,
creating strains on the community of developers and maintainers of a
standard who will need to accommodate long deprecation cycles. On the
other hand, in cases in which some forms of dynamic change is prohibited
-- as in the case of the FITS file format, which prohibits changes that
break backwards-compatibility -- there is also a cost associated with
the stability \citep{Scroggins2020-ut}: limiting adoption and
combinations of new types of measurements, new analysis methods or new
modes of data storage and data sharing.

\subsection{Mismatches between standards developers and user
communities}\label{mismatches-between-standards-developers-and-user-communities}

Open-source standards often entail an inherent gap between the core
developers of the standard and the users of the standard. The former may
be possess higher ability to engage with the technical details
undergirding standards and their development, while the latter still
have a high level of interest as members of the broader research field
to which the standard pertains. This gap, in and of itself, creates
friction on the path to broad adoption and best utilization of the
standards. In extreme cases, the interests of researchers and standards
developers may even seem at odds, as developers implement sophisticated
mechanisms to automate the creation and validation of the standard or
advocate for more technically advanced mechanisms for evolving the
standard. These advanced capabilities offer more robust development
practices and consistency in cases where the standards are complex and
elaborate. They can also ease the maintenance burden of the standard. On
the other hand, they may end up leaving potential experimental
researchers and data providers sidelined in the development of the
standard, and limiting their ability to provide feedback about the
practical implications of changes to the standards. One example of this
(already mentioned above in Section~\ref{sec-use-cases}) is the use of
git/GitHub for versioning of standards documents. This sets a high bar
for participation in standards development for researchers in fields of
research in which git/GitHub have not yet had significant adoption as
tools of day-to-day computational practice. At the same time, it
provides clarity and robustness for standards developers communities
that are well-versed in these tools.

Another layer of potential mismatches arises when a more complex set of
stakeholders needs to be considered. For example, the Group on Earth
Observations (GEO) is a network that aims to coordinate decision making
around satellite missions and to standardize the data that results from
these missions. Because this group involves a range of different
stakeholders, including individuals who more closely understand
potential legal issues and researchers who are better equipped to
evaluate technical and domain questions, communication is slower and
hindered. As the group aims to move forward by consensus, these
communication difficulties can slow down progress. This is just an
example, which exemplifies the many cases in which OSS process which
strives for consensus can slow progress.

\subsection{Cross-domain gaps}\label{cross-domain-gaps}

There is much to be gained from the development of standards that apply
in multiple different domains. For example, many research fields use
images as data and array-based computing that is applicable to images in
various research fields is at the core of many scientific computing
codes. This means that practitioners within any given field should be
motivated to draw on shared data standards and shared software
implementations of operations that are common across fields. On the
other hand, it is very hard to justify the investment in these common
resources. On the one hand, data standardization investment is even more
justified if the standard is generalizable beyond any specific science
domain. On the other hand, while the use cases are domain sciences
based, data standardization is seen as a data infrastructure and not a
science investment, reducing the immediate incentives for researchers to
engage with such efforts. This is exacerbated by science research
funding schemes that eschew generalized cross-domain solutions, and that
seek to generate tangible impact only with a specific domain.

\subsection{Data instrumentation
issues}\label{data-instrumentation-issues}

Where there is commercial interest in the development of data analysis
tools (e.g., in biomedical applications or applications were research
funding can be directed towards commercial solutions) there is an
incentive to create data formats and data analysis platforms that are
proprietary. This may drive innovative applications of scientific
measurements, but also creates sub-fields where scientific observations
are generated by proprietary instrumentation, due to these
commercialization or other profit-driven incentives. FTIR Spectroscopy
is one such example, wherein use of Bruker instrumentation necessitates
downstream analysis of the resulting measurements using proprietary
binary formats necessary for the OPUS Software. Another example is the
proliferation of proprietary file formats in electrophysiological
measurements of brain signals \citep[@Hermes2023-aw]{Gillon2024-vu}. And
yet another one is proprietary application programming interfaces (APIs)
used in electronic health records
\citep[@Adler-Milstein2017-id]{Barker2024-ox}. In most cases, there is a
lack of regulatory oversight to adhere to available standards or evolve
common tools, limiting integration across different measurements. In
cases where a significant amount of data is already stored in
proprietary formats, or where access is limited by proprietary APIs
significant data transformations may be required to get data to a state
that is amenable to open-source standards. In these sub-fields there may
also be a lack of incentive to set aside investment or resources to
invest in establishing open-source data standards, leaving these
sub-fields relatively siloed.

\subsubsection{Harnessing new computing paradigms and
technologies}\label{harnessing-new-computing-paradigms-and-technologies}

Open-source standards development faces the challenges of adapting to
new computing paradigms and technologies. Cloud computing provides a
particularly stark set of opportunities and challenges. On the one hand,
cloud computing offers practical solutions for many challenges of
contemporary data-driven research. For example, the scalability of cloud
resources addresses some of the challenges of the scale of data that is
produced by instruments in many fields. The cloud also makes data access
relatively straightforward, because of the ability to determine data
access permissions in a granular fashion. On the other hand, cloud
computing requires reinstrumenting many data formats. This is because
cloud data access patterns are fundamentally different from the ones
that are used in local posix-style file-systems. Suspicion of cloud
computing comes in two different flavors: the first by researchers and
administrators who may be wary of costs associated with cloud computing,
and especially with the difficulty of predicting these costs. This can
particularly affect scenarios where long-term preservation is required.
Projects such as NSF's Cloud Bank seek to mitigate some of these
concerns, by providing an additional layer of transparency into cloud
costs \citep{Norman2021CloudBank}. The other type of objection relates
to the fact that cloud computing services, by their very nature, are
closed ecosystems that resist portability and interoperability. Some
aspects of the services are always going to remain hidden and privy only
to the cloud computing service provider. In this respect, cloud
computing runs afoul of some of the appealing aspects of OSS. That said,
the development of ``cloud native'' standards can provide significant
benefits in terms of the research that can be conducted. For example,
NOAA plans to use cloud computing for integration across the multiple
disparate datasets that it collects to build knowledge graphs that can
be queried by researchers to answer questions that can only be answered
through this integration. Putting all the data ``in one place'' should
help with that. Adaptation to the cloud in terms of data standards has
driven development of new file formats. A salient example is the ZARR
format \citep{zarr}, which supports random access into array-based
datasets stored in cloud object storage, facilitating scalable and
parallelized computing on these data. Indeed, data standards such as NWB
(neuroscience) and OME (microscopy) now use ZARR as a backend for
cloud-based storage. In other cases, file formats that were once not
straightforward to use in the cloud, such as HDF5 and TIFF have been
adapted to cloud use (e.g., through the cloud-optimized geoTIFF format).

\subsection{Unclear pathways for standards success and
sustainability}\label{unclear-pathways-for-standards-success-and-sustainability}

The development of open-source standards faces similar sustainability
challenges to those faced by open-source software that is developed for
research. Standards typically develop organically through sustained and
persistent efforts from dedicated groups of data practitioners. These
include scientists and the broader ecosystem of data curators and users.
However, there is no playbook on the structure and components of a data
standard, or the pathway that moves the implementation of a specific
data architecture (e.g., a particular file format) to become a data
standard. As a result, data standardization lacks formal avenues for
success and recognition, for example through dedicated research grants
(and see Section~\ref{sec-cross-sector}). This hampers the long-term
trajectory that is needed to inculcate a standard into the day-to-day
practice of researchers.

\section{Cross-sector interactions}\label{sec-cross-sector}

The importance of standards stems not only from discussions within
research fields about how research can best be conducted to take
advantage of existing and growing datasets, but also arises from
interactions with stakeholders in other sectors. Several different kinds
of cross-sector interactions can be defined as having an important
impact on the development of open-source standards.

\subsection{Governmental
policy-setting}\label{governmental-policy-setting}

The development of open practices in research has entailed an ongoing
interaction and dialogue with various governmental bodies that set
policies for research. For example, for research that is funded by the
public, this entails an ongoing series of policy discussions that
address the interactions between research communities and the general
public. One way in which this manifests in the United States
specifically is in memos issued by the directors of the White House
Office of Science and Technology Policy (OSTP), James Holdren (in 2013)
and Alondra Nelson (in 2022). While these memos focused primarily on
making peer-reviewed publications funded by the US Federal government
available to the general public, they also lay an increasingly detailed
path toward the publication and general availability of the data that is
collected in research that is funded by the US government. The general
guidance and overall spirit of these memos dovetail with more specific
policy guidance related to data and metadata standards. For example, the
importance of standards was underscored in a recent report by the
Subcommittee on Open Science of the National Science and Technology
Council on the ``Desirable characteristics of data repositories for
federally funded research'' \citep{nstc2022desirable}. The report
explicitly called out the importance of ``allow{[}ing{]} datasets and
metadata to be accessed, downloaded, or exported from the repository in
widely used, preferably non-proprietary, formats consistent with
standards used in the disciplines the repository serves.'' This
highlights the need for data and metadata standards across a variety of
different kinds of data. In addition, a report from the National
Institute of Standards and Technology on ``U.S. Leadership in AI: A Plan
for Federal Engagement in Developing Technical Standards and Related
Tools'' emphasized that -- specifically for the case of AI -- ``U.S.
government agencies should prioritize AI standards efforts that are
{[}\ldots{]} Consensus-based, {[}\ldots{]} Inclusive and accessible,
{[}\ldots{]} Multi-path, {[}\ldots{]} Open and transparent, {[}\ldots{]}
and {[}that{]} result in globally relevant and non-discriminatory
standards\ldots{}'' \citep{NIST2019}. The converging characteristics of
standards that arise from these reports suggest that considerable
thought needs to be given to how standards arise so that these goals are
achieved. Importantly, open-source standards seem to well-match at least
some of these characteristics.

The other side of policies is the implementation of these policies in
practice by developers of open-source standards and by the communities
to which the standards pertain. A compelling road map towards
implementation and adoption of open science practices in general and
open-source standards in particular is offered in a blog post authored
by the Center for Open Science's co-founder and executive director,
Brian Nosek, entitled ``Strategy for Culture Change''
\citep{Nosek2019CultureChange}. The core idea is that affecting a turn
toward open science requires an alignment of not only incentives and
values, but also technical infrastructure and user experience. A
sociotechnical bridge between these pieces, which makes the adoption of
standards possible, and maybe even easy, and the policy goals, arises
from a community of practice that makes the adoption of standards
\emph{normative}. Once all of these pieces are in place, making adoption
of open science standards \emph{required} through policy becomes more
straightforward and less onerous.

\subsection{Funding}\label{funding}

Government-set policy intersects with funding considerations. This is
because it is primarily directed towards research that is funded through
governmental funding agencies. For example, high-level policy guidance
boils to practice in guidance for data management plans that are part of
funded research. In response to the policy guidance, these have become
increasingly more detailed and, for example, NSF- and NIH-funded
researchers are now required to both formulate their plans with more
clarity and increasingly also to share data using specified standards as
a condition for funding.

However, there are other ways in which funding relates to the
development of open-source standards. For example, through the BRAIN
Initiative, the National Institutes of Health have provided key funding
for the development of the Brain Imaging Data Structure standard in
neuroscience. Where large governmental funding agencies may not have the
resources or agility required to fund nascent or unconventional ways of
formulating standards, funding by non-governmental philanthropies and
other organizations can provide alternatives. One example (out of many)
is the Chan-Zuckerberg Initiative program for Essential Open Source
Software, which funds foundational open-source software projects that
have an application in biomedical sciences. Distinct from NIH funding,
however, some of this investment focuses on the development of OSS
practices. For example, funding to the Arrow project that focuses on
developing open-source software maintenance skills and practices, rather
than a specific biomedical application.

\subsection{Industry}\label{industry}

Interactions of data and meta-data standards with commercial interests
may provide specific sources of friction. This is because
proprietary/closed formats of data can create difficulty at various
transition points: from one instrument vendor to another, from data
producer to downstream recipient/user, etc. On the other hand, in some
cases, cross-sector collaborations with commercial entities may pave the
way to robust and useful standards. For example, imaging measurements in
human subjects (e.g., in brain imaging experiments) significantly
interact with standards for medical imaging, and chiefly the Digital
Imaging and Communications in Medicine (DICOM) standard, which is widely
used in a range of medical imaging applications, including in clinical
settings \citep[@Mustra2008-xk]{Larobina2023-vq}. The standard emerged
from the demands of the clinical practice in the 1980s, as digital
technologies were came into widespread use in medical imaging, through
joint work of industry organizations: the American College of Radiology
and the National Association of Electronic Manufacturers. One of the
defining features of the DICOM standard is that it allows manufacturers
of instruments to define ``private fields'' that are compliant with the
standard, but which may include idiosyncratically organized data and/or
metadata. This provides significant flexibility, but can also easily
lead to the loss of important information. Nevertheless, the human brain
imaging case is exemplary of a case in which industry standards and
research standards coexist and need to communicate with each other
effectively to advance research use-cases, while keeping up with the
rapid development of the technologies.

\section{Recommendations for open-source data and metadata
standards}\label{sec-recommendations}

In conclusion of this report, we would like to propose a set of
recommendations that distill the lessons learned from an examination of
data and metadata standards through the lense of open-source software
development practices. We divide this section into two parts: one aimed
at the science and technology communities that develop and maintain
open-source standards, and the other aimed at policy-making and funding
agencies, who have an interest in fostering more efficient, more robust,
and more transparent open-source standards.

\subsection{Science and technology
communities:}\label{science-and-technology-communities}

\subsubsection{Establish standards governance based on OSS best
practices}\label{establish-standards-governance-based-on-oss-best-practices}

While best-practice governance principles are also relatively new in OSS
communities, there is already a substantial set of prior art in this
domain, on which the developers and maintainers of open-source data and
metadata standards can rely. For example, it is now clear that
governance principles and rules can mitigate some of the risks and
challenges mentioned in Section~\ref{sec-challenges}, especially for
communities beyond a certain size that need to converge toward a new
standard or rely on an existing standard. Developers and maintainers
should review existing governance practices such as those provided by
The Open Source
Way(\href{https://www.theopensourceway.org/the_open_source_way-guidebook-2.0.html\#_project_and_community_governance}{https://www.theopensourceway.org/}).

\subsubsection{Foster meta-standards
development}\label{foster-meta-standards-development}

One of the main conclusions that arise from our survey of the landscape
of existing standards is that there is significant knowledge that exists
across fields and domains and that informs the development of standards
within each field, but that could be surfaced to the level where it may
be adopted more widely in different domains and be more broadly useful.
One approach to this is a comparative approach: in this approach, a
readiness and/or maturity model can be developed that assesses the
challenges and opportunities that a specific standard faces at its
current phase of development. Developing such a maturity model, while it
goes beyond the scope of the current report, could lead to the eventual
development of a meta-standard or a standard-of-standards. This would
facilitate a succinct description of cross-cutting best-practices that
can be used as a basis for the analysis or assessment of an existing
standard, or as guidelines to develop new standards. For instance,
specific barriers to adopting a data standard that take into account the
size of the community and its specific technological capabilities should
be considered.

More generally, meta-standards could include formalization for
versioning of standards and interactions with specific related software.
This includes amplifying formalization/guidelines on how to create
standards (for example, metadata schema specifications using LinkML,
\url{https://linkml.io}). However, aspects of communication with
potential user audiences (e.g., researchers in particular domains)
should be taken into account as well. For example, in the quality of
onboarding documentation and tools for ingestion or conversion into
standards-compliant datasets.

An ontology for the standards-development process -- for example
top-down vs bottom-up, minimum number of datasets, target community size
and technical expertise typical of this community, and so forth -- could
help guide the standards-development process towards more effective
adoption and use. A set of meta-standards and high-level descriptions of
the standards-development process -- some of which is laid out in this
report -- could help standard developers avoid known pitfalls, such as
the dreaded proliferation of standards, or complexity-impeded adoption.
Surveying and documenting the success and failures of current standards
for a specific dataset / domain can help disseminate knowledge about the
standardization process. Resources such as Fairsharing (
\url{https://fairsharing.org/}) or the Digital Curation Center
(\url{https://www.dcc.ac.uk/guidance/standards}) can help guide this
process.

\subsubsection{Develop standards in tandem with standards-associated
software}\label{develop-standards-in-tandem-with-standards-associated-software}

Development of standards should be coupled and tightly linked with
development of associated software. This produces a virtuous cycle where
the use-cases and technical issues that arise in software development
informs the development of the standard and vice versa. One of the
lessons learned across a variety of different standards is the
importance of automated validation of the standard. Automated validation
is broadly seen as a requirement for the adoption of a standard and a
factor in managing change of the standard over time. To advance this
virtuous cycle, we recommend to make data standards machine readable,
and make software creation an integral part of establishing a standard's
schema. Additionally, standards evolution should maintain software
compatibility, and ability to translate and migrate between standards.

\subsection{Policy-making and funding
entities:}\label{policy-making-and-funding-entities}

\subsubsection{Fund the development of open-source
standards}\label{fund-the-development-of-open-source-standards}

While some funding agencies already support standards development as
part of the development of informatics infrastructures, data standards
development should be seen as integral to science innovation and
earmarked for funding in research grants, not only in specialized
contexts. Funding models should encourage the development and adoption
of standards, and fund associated community efforts and tools for this.
The OSS model is seen as a particularly promising avenue for an
investment of resources, because it builds on previously-developed
procedures and technical infrastructure and because it provides avenues
for the democratization of development processes and for community input
along the way. At the same time, there are significant challenges
associated with incentives to engage, ranging from the dilution of
credit to individual contributors, and ranging through the burnout of
maintainers and developers. The clarity offered by procedures for
enhancement proposals and semantic versioning schemes adopted in
standards development offers avenues for a range of stakeholders to
propose well-defined contributions to large and field-wide standards
efforts (e.g., \citep{pestilli2021community}), and potentially helps
alleviate some of these concerns by providing avenues for individual
contributions to surface, as well as clarity of process, which can
alleviate the risks of maintainer burnout.

\subsubsection{Invest in data stewards}\label{invest-in-data-stewards}

Advancing the development and adoption of open-source standards requires
the dissemination of knowledge to researchers in a variety of fields,
but this dissemination itself may not be enough without the fostering of
specialized expertise. Therefore, it is important to recognize the
distinct role that \emph{data stewards} play in contemporary research.
As policy demands for openness become increasingly high, it is crucial
to truly support experts whose role will be to develop, maintain, and
facilitate the adoption and use of open-source standards. This support
needs to manifest in all stages of the career of these individuals: it
will be necessary to set up programs for training for data stewards, and
to invest in the career paths of individuals that receive such training
so that this crucial role is encouraged. Initial proposals for the
curriculum and scope of the role have already been proposed (e.g., in
\citep{Mons2018DataStewardshipBook}), but we identify here also a need
to connect these individuals directly to the practices that exemplify
open-source standards. Thus, it will be important for these individuals
to be conversant in the methodology of OSS. This does not mean that they
need to become software engineers -- though for some of them there may
be some overlap with the role of research software engineers
\citep{Connolly2023Software} -- but rather that they need to become
familiar with those parts of the OSS development life-cycle that are
specifically useful for the development of open-source standards. For
example, tools for version control, tools for versioning, and tools for
creation and validation of compliant data and metadata. Stakeholder
organizations should invest in training grants to establish curriculum
for data and metadata standards education.

Ultimately, efficient use of data stewards and their knowledge will have
to be applied. It is evident that not every project and every lab that
produces data requires a full-time data steward. Instead, data
stewardship could be centralized within organizations such as libraries,
data science, or software engineering cores of larger research
organizations. This would be akin to recent models for research software
engineering that are becoming common in many research organization
\citep{Van-Tuyl2023-vp}. Efficiency considerations also suggest that the
development of data standards would not have its intended purpose unless
funds are also allocated to the implementation of the standard in
practice. Mandating standards without appropriate funding for their
implementation by data producers and data users could risk hampering
science and could leading to researchers doing the bare minimum to make
their data ``open''.

\subsubsection{Review open-source standards
pathways}\label{review-open-source-standards-pathways}

Invest in programs that examine retrospective pathways for establishing
data standards. Encourage publication of lifecycles for successful data
standards. These lifecycles should include the process, creators,
affiliations, grants, and adoption journeys of open-source standards. To
encourage sustainable development of open-source standards, and to build
on prior experience, the documentation and dissemination of lifecycles
should be seen as an integral step of the work of standards creators and
granting agencies. In the meanwhile, it would be good to also
retroactively document the lifecycle of existing standards that are seen
as success stories, and to foster the awareness of these standards. In
addition, fostering research projects on the principles that underlie
successful open-source standards development will help formulate new
standards and iterate on existing ones. In accordance, data management
plans should promote the sharing of not only data, but also metadata and
descriptions of how to use it.

\subsubsection{Manage Cross Sector
alliances}\label{manage-cross-sector-alliances}

Encourage cross-sector and cross-domain alliances that can impact
successful standards creation. Invest in robust program management of
these alliances to align pace and create incentives (for instance via
Open Source Program Offices at Universities or other research
organizations). Similar to program officers at funding agencies,
standards evolution need sustained PM efforts. Multi-party partnerships
should include strategic initiatives for standard establishment such as
the Pistoia Alliance (\url{https://www.pistoiaalliance.org/}).

\section{Acknowledgements}\label{acknowledgements}

This report was produced following a
\href{https://uwescience.github.io/2024-open-source-standards-workshop/}{workshop
held at NSF headquarters in Alexandria, VA on April 8th-9th, 2024}. We
would like to thank the speakers and participants in this workshop for
the time and thought that they put into the workshop. A list of workshop
participants is provided as an appendix (Section~\ref{sec-appendix}).

The workshop and this report were funded through
\href{https://www.nsf.gov/awardsearch/showAward?AWD_ID=2334483&HistoricalAwards=false}{NSF
grant \#2334483} from the NSF
\href{https://new.nsf.gov/funding/opportunities/pathways-enable-open-source-ecosystems-pose}{Pathways
to Enable Open-Source Ecosystems (POSE)} program. The opinions expressed
in this report do not necessarily reflect those of the National Science
Foundation.

\section{References}\label{references}

\renewcommand{\bibsection}{}
\bibliography{references.bib}

\newpage

\section{Appendix: List of participants}\label{sec-appendix}

\begin{longtable}[]{@{}ll@{}}
\toprule\noalign{}
Name & Affiliation \\
\midrule\noalign{}
\endhead
\bottomrule\noalign{}
\endlastfoot
Alex D Wade & Digital Science \\
Alexander Szalay & Johns Hopkins University \\
Andrew Connolly & University of Washington \\
Anshul Tushar Tambay & University of Washington \\
Ariel Rokem & University of Washington \\
Carolina Lorena Berys & University of California, San Diego \\
Christine Kirkpatrick & San Diego Supercomputer Center \\
Fernando Seabra Chirigati & Nature Computational Science \\
Jessica Morgan & NOAA \\
John Relph & NOAA \\
Julia Ferraioli & Open Source Stories \\
Jurriaan Hein Spaaks & formerly Netherlands eScience Center \\
Justin (Jay) Hnilo & Department of Energy \\
Kalynn Elisabeth Kennon & Infectious Diseases Data Observatory \\
Kevin Christopher Booth & Radiant Earth \\
Kristofer E. Bouchard & Lawrence Berkeley National Labs \\
Lea A. Shanley & University of California, Berkeley \\
Michael Spannowsky & Durham University \\
Nicoleta C Cristea & University of Washington \\
Nina Amla & NSF \\
Oliver Ruebel & Lawrence Berkeley National Labs \\
Ray E. Habermann & Metadata Game Changers \\
Raymond (Ray) Plante & NIST \\
Robert Hanisch & NIST \\
Saskia de Vries & Allen Institute for Neural Dynamics \\
Steven Crawford & NASA \\
Vani Mandava & University of Washington \\
Yaroslav Halchenko & Dartmouth University \\
Ziheng Sun & George Mason University \\
\end{longtable}





\end{document}
